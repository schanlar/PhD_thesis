\documentclass[main.tex]{subfiles}

\begin{document}
    \chapter[Summary and Future Work]{Summary and Future Work}
    In this chapter, we summarize the key findings of the thesis, providing a comprehensive overview of the major conclusions drawn from the research. Furthermore, we discuss the open questions and future directions that emerge from this research. Despite the valuable insights provided, several unresolved issues remain, which could be explored in subsequent studies. These include areas where further investigation is needed to clarify ambiguities, expand on the findings, or address new questions that arose during the course of the research.

    {\hypersetup{linkcolor=black, pdfborder={0 0 0}}
        \minitoc
        \newpage
    }
    
    \section{Overview} \label{sec:ch5:overview}
    The first study modeled the evolution of helium stars across a wide range of masses and metallicities, examining the conditions under which they collapse into neutron stars via electron-capture supernovae (ECSNe) or disrupt in thermonuclear explosions as type Ia supernovae (SNe Ia). A critical determinant of the outcome was the residual carbon retained in the core after burning phases. Stars with near-Chandrasekhar mass cores that retain a moderate amount of carbon ignite oxygen at relatively low densities, triggering thermonuclear explosions that produce ejecta with kinetic energies comparable to classical SNe Ia. This mechanism, which does not rely on accretion, suggests that stripped helium stars could contribute to the observed diversity of SNe Ia, particularly in young, star-forming galaxies. In contrast, helium stars with lower residual carbon followed paths toward ECSNe. Additionally, metallicity was shown to play a key role, influencing the relative rates of thermonuclear and electron-capture events.

    Building on this foundation, the second study explored the impact of residual hydrogen on the evolution of partially stripped helium stars. Unlike fully stripped stars, these stars retained thin hydrogen envelopes after binary interactions. The presence of residual hydrogen was found to significantly alter stellar evolution by affecting core composition and shell-burning dynamics. Stars with substantial hydrogen, approximately $0.5\msun$, developed oxygen-neon (ONe) white dwarfs containing residual carbon, making them susceptible to later accretion-induced thermonuclear explosions. In contrast, stars with minimal hydrogen evolved similarly to their fully stripped counterparts, potentially leading to thermonuclear or electron-capture supernovae. These findings highlight the complexity introduced by incomplete stripping in binary systems.

    The third study introduced a distinct perspective by investigating twin compact stars in low-mass X-ray binaries. It showed that phase transitions in neutron star cores during accretion could lead to the formation of hybrid neutron stars composed of a quark core. These transitions affect binary dynamics, potentially producing eccentric millisecond pulsars (MSPs) or isolated MSPs. This study offers valuable insight into neutron star structure, the role of phase transitions, and their observational signatures in binary orbital parameters.
    
    \section{Future Directions and Open Questions} \label{sec:ch5:future}
    While the findings presented in this thesis represent significant advances, they also highlight key areas where further exploration is necessary:
    \begin{enumerate}
        \item \textbf{Improved Modeling of Mass Loss and Binary Interactions}\\
        One critical avenue involves improving the modeling of mass loss and binary interactions, which remain key uncertainties in determining the final fates of helium stars. Current models use simplified prescriptions for mass loss, especially during common-envelope phases or wind-driven stripping. Multidimensional simulations that account for episodic mass loss, detailed hydrodynamic processes, and angular momentum transport could provide a more complete picture, particularly for stars poised between thermonuclear explosion and collapse.

        \item \textbf{Observational Signatures of (C)ONe SNe-Ia}\\
        Another pressing question concerns the observational signatures of (C)ONe SNe-Ia and the exact contributions of stripped helium stars to the observed diversity of SNe Ia in terms of observable properties, such as spectra, light curves, and ejecta composition.
        Developing predictive tools that connect progenitor core structures with post-explosion properties could enable observational surveys to identify these events more confidently, thereby clarifying their contributions to supernova rates. In addition, studying their remnants---such as the compact objects or unique isotopic signatures left behind---would further aid in understanding their roles in stellar evolution.
        Moreover, the explored mechanisms suggest that stripped helium stars could account for a non-negligible fraction of SNe Ia, particularly in environments with active star formation. Future research should estimate the frequency of these events in comparison to other progenitor channels, considering metallicity dependence and binary population synthesis models.

        \item \textbf{Impact of Residual Hydrogen on Accretion-Induced Collapse}\\
        The influence of residual hydrogen in shaping the structure of ONe white dwarfs and their subsequent evolution requires further study. Multidimensional simulations could explore the precise conditions under which residual carbon triggers accretion-induced collapse or thermonuclear runaway

        \item \textbf{Physics of Dense Matter}\\
        The study of compact objects and dense matter physics offers a wide range of opportunities for further exploration, driven by advances in theoretical modeling and observational techniques. One major area of focus is the physics of first-order phase transitions in neutron star cores. These transitions, which can significantly alter a star’s internal structure, are predicted to influence binary dynamics and lead to phenomena such as eccentric millisecond pulsars or isolated compact stars. Future research should expand numerical simulations to better understand the observable consequences, including gravitational wave signatures and orbital characteristics.
        
    \end{enumerate}
    

\end{document}