\documentclass[main.tex]{subfiles}

\begin{document}
    \chapter*{Abstract}
    Neutron stars (NSs) are compact remnants of massive stars formed in supernova explosions, with properties intricately linked to the evolution of their progenitors and the characteristics of the explosion. This thesis investigates the structure and evolution of neutron stars, beginning with an overview of stellar astrophysics and the processes leading to compact object formation. Special emphasis is placed on the complex internal structure of neutron stars, particularly the challenges posed by the extreme conditions found within their cores. It then presents three separate research papers in subsequent chapters.

    Through detailed numerical modeling, the first research paper examines neutron stars formed from the collapse of intermediate-mass stars through electron-capture supernovae (ECSN). Stars with initial masses of $7–11\msun$ that lose their hydrogen envelopes may experience outcomes ranging from ECSN to thermonuclear supernovae, depending on their final core composition and structure. Modeling of stripped, helium stars reveals that cores around $1.35–1.37\msun$ can initiate explosive oxygen burning at relatively low densities, leading to thermonuclear explosions if residual carbon exceeds $\sim 0.005\msun$, while lower carbon content favors ECSN. The study further finds that (C)ONe Type Ia supernovae are more likely at higher metallicities.

    The second research paper investigates the evolutionary outcomes of massive stars in binary systems, focusing on the impact of incomplete mass stripping on their final fate. In such systems, the donor star may retain a thin hydrogen envelope, significantly influencing its evolution. Through numerical modeling of partially stripped helium stars at solar metallicity, the study demonstrates that a hydrogen mass as small as $0.5\msun$ can sustain hot bottom burning, accelerating evolution and producing oxygen-neon white dwarfs with residual carbon in their cores. Conversely, minimal hydrogen retention leads to explosive oxygen burning. These results highlight how inefficient stripping in intermediate-mass stars can naturally lead to oxygen-neon white dwarfs prone to thermonuclear runaway during later mass transfer, potentially disrupting neutron star formation via accretion-induced collapse.

    The third research paper explores the formation of twin compact stars through rapid phase transitions in NS cores due to mass accretion in low-mass X-ray binaries (LMXBs).
    Neutron stars in binary systems can experience episodes of mass accretion from their companion stars. During such mass transfer episodes, the neutron star’s mass---and therefore its central density---gradually increases. As the core density rises, it may reach a threshold where a phase transition from hadronic matter to deconfined quark matter can occur. This phase transition marks a fundamental shift from a purely hadronic composition, which is dominated by neutrons, protons, and electrons, to a more exotic state where quarks are no longer confined within individual nucleons, yielding a quark core. Simulations reveal that phase transitions may occur during the LMXB phase in compact binaries or during spin-down in wider systems, possibly leading to eccentric binary millisecond pulsars (MSPs). If these transitions involve secondary kicks exceeding 20 km/s, they may disrupt the binary, forming isolated MSPs or reconfiguring them to ultra-wide orbits. The eccentricity distribution of binary MSPs could provide constraints on phase transitions in dense nuclear matter.

    The thesis concludes with a comprehensive summary of the key findings and insights derived from the research conducted. In this final section, the main conclusions drawn from the study are highlighted, emphasizing their significance in the broader context of the field. Additionally, a discussion is provided reflecting on any limitations of the current study and suggesting areas for further investigation.

\end{document}