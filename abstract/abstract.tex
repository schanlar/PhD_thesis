\documentclass[main.tex]{subfiles}

\begin{document}
    \chapter*{Abstract}
    Neutron stars (NSs) are compact remnants of massive stars formed in supernova explosions, with properties intricately linked to fundamental physical laws, the evolution of their progenitors, and the characteristics of the explosion. This thesis investigates the structure and evolution of neutron stars, beginning with an overview of stellar astrophysics and the processes leading to compact object formation. Special emphasis is placed on the complex internal structure of neutron stars, particularly the challenges posed by the extreme conditions found within their cores. It then presents two separate research papers in subsequent chapters,

    Through detailed numerical modeling, this study examines neutron stars formed from the collapse of intermediate-mass stars through electron-capture supernovae (ECSN). Stars with initial masses of $7–11\msun$ that lose their hydrogen envelopes may experience outcomes ranging from ECSN to thermonuclear (C)ONe Type Ia supernovae, depending on their final core composition and structure. Modeling of stripped, helium stars reveals that cores around $1.35–1.37\msun$ can initiate explosive oxygen burning at specific densities, leading to thermonuclear explosions if residual carbon exceeds $\sim 0.005\msun$, while lower carbon content favors ECSN. The study further finds that (C)ONe Type Ia supernovae are more likely at higher metallicities, with late-stage expansion possibly triggering binary interactions in both progenitor types.

    Additionally, this thesis explores the formation of twin compact stars through rapid phase transitions in NS cores due to mass accretion in low-mass X-ray binaries (LMXBs). Simulations reveal that phase transitions may occur during the LMXB phase in compact binaries or during spin-down in wider systems, possibly leading to eccentric binary millisecond pulsars (MSPs). If these transitions involve secondary kicks exceeding 20 km/s, they may disrupt the binary, forming isolated MSPs or reconfiguring them to ultra-wide orbits. The eccentricity distribution of binary MSPs could provide constraints on phase transitions in dense nuclear matter.

    The thesis concludes with a summary of the main conclusions and a discussion on the ramifications of this work.



    

    % Using novel numerical modeling, this work examines neutron stars formed from the collapse of intermediate-mass stars through electron-capture supernovae (ECSN), revealing that some stars may instead undergo thermonuclear runaway, preventing NS formation. This analysis further constrains the conditions necessary for such outcomes and explores their implications for the observed distribution of NS masses.

    % Additionally, this thesis models the evolution of neutron stars within binary systems, exploring phase transitions induced by mass accretion that could result in hybrid stars with exotic matter compositions.
    
    % Neutron stars (NSs) are the compact remnants of massive stars formed in supernova explosions. Their properties depend sensitively on the properties of fundamental physical laws, the evolution of their progenitors, and the physical characteristics of the explosion. This thesis, [...]

    % Chapter~1 provides a brief and necessary introduction to stellar astrophysics and is divided into two parts; In the first part, I make an attempt to equip the reader with the essentials of stellar structure and evolution that leads to the formation of compact objects. In the second part, I focus on neutron stars where I examine their structure and the complexities arising with extreme dense matter found in their interior.
 

\end{document}