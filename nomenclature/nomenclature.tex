\documentclass[main.tex]{subfiles}

\begin{document}
\chapter*{Glossary and Nomenclature}

\section*{Frequently Used Symbols}
\textbf{Physical Quantities}
\begin{longtable}{p{0.1\textwidth} p{0.8\textwidth}}
    $A$ & Mass number: It refers to the total number of protons and neutrons in an atomic nucleus. \\\\
    $a$ & Semi-major axis: It is the longest radius of an ellipse, representing half the longest diameter in the elliptical orbit of a celestial body.  \\\\
    $B$ & Magnetic flux density: It measures the strength and direction of the magnetic field. \\\\
    $e$ & \inlineitem Eccentricity: It describes the deviation of an ellipse from a perfect circle; \inlineitem Electric charge: Refers to the elementary electric charge, which is the smallest unit of electric charge that is observed in nature, approximately equal to $1.602 \times 10^{-19}\,\text{C}$; \inlineitem Numerical constant: Base of the natural logarithm, approximately equal to $2.71828$. \\\\
    $E_\mathrm{gr}$ & Gravitational energy: It is the potential energy a body with mass has in a gravitational field. \\\\
    $E_\mathrm{kin}$ & Kinetic energy: It is the energy that an object possesses due to its motion. \\\\
    \resetinlineenum
    $g$ & \inlineitem Degeneracy: It refers to the quantum states of a system where several configurations have the same energy level; \inlineitem Gravitational acceleration: The acceleration experienced by an object due to the gravitational force exerted by a massive body. \\\\
    $J$ & Orbital angular momentum: It is the angular momentum of an object in orbit around another object. \\\\
    $L$ & Luminosity: It is the total amount of energy emitted by a star, galaxy, or other astronomical object per unit time. \\\\
    $M$, $m$ & Gravitational mass: It s the property of an object that determines the strength of its gravitational interactions with other masses in accordance with Newton's law of universal gravitation. \\\\
    $M_b$ & Baryonic mass: It refers to the sum of mass contributed by baryons, such as protons and neutrons, in an object, representing the conventional matter content. \\\\
    $M_\mathrm{Ch}$ & Chandrasekhar mass, $M_\mathrm{Ch} \approx 1.4\ M_{\odot}$: The maximum mass of a stable white dwarf star, beyond which it would collapse into a neutron star or black hole. \\\\
    $M_J$ & Jeans mass: It is the critical mass where gravitational collapse overcomes pressure forces to form a star. \\\\
    \resetinlineenum
    $n$ & \inlineitem Number density: It is the concentration of countable objects in physical space, such as particles per unit volume; \inlineitem Neutron: A subatomic particle found in the nucleus of an atom, electrically neutral with no net charge, and has a mass slightly greater than that of a proton. \\\\
    \resetinlineenum
    $P$ & \inlineitem Pressure: It is the force applied perpendicular to the surface of an object per unit area; \inlineitem  Orbital period: Refers to the duration required for a celestial body to complete a single orbit around another celestial body or focal point. \\\\
    \resetinlineenum
    $p$ & \inlineitem Momentum: Represents the product of an object's mass and velocity, indicating the direction and magnitude of its motion; \inlineitem Proton: A subatomic particle found in the nucleus of an atom with a positive electric charge. \\\\
    $Q$ & Energy release in nuclear reactions: It quantifies the energy gained or lost during a nuclear reaction. \\\\
    $q$ & Mass ratio: It is the ratio of the mass of one body to another. \\\\
    \resetinlineenum
    $R$ & \inlineitem Radius: It refers to the distance from the center to the circumference of a circle or sphere; \inlineitem Scalar curvature: A scalar value that describes the curvature of a space at a point. \\\\
    $R_L$ & Roche lobe radius: It is the region around a star in a binary system within which orbiting material is gravitationally bound to that star. \\\\
    $r$ & Distance (or radius): It measures the space between two points. \\\\
    $r_s$ & Schwarzschild radius: The radius at which the escape velocity equals the speed of light, which marks the event horizon of a black hole. \\\\
    $T$ & Temperature: It measures the thermal energy of a substance. \\\\
    $T_\mathrm{eff}$ & Effective temperature: It is the temperature of a black body that would emit the same total amount of electromagnetic radiation. \\\\
    $t$ & Time: A non-spatial continuum in which events occur in irreversible succession from the past, through the present, to the future, serving as a measure of durations and intervals between events. \\\\
    $u$, $\varepsilon$ & Energy density: It is the amount of energy stored in a given system or region of space per unit volume. \\\\
    $V$ & Volume: It is the quantity of three-dimensional space enclosed by a closed surface. \\\\
    $v$, $w$ & Velocity: It is the speed of an object in a given direction. \\\\
    \resetinlineenum
    $X$ & \inlineitem Fraction of hydrogen: It is the proportion of hydrogen in a mixture or celestial body; \inlineitem Average mass fraction of an isotope: The ratio of the mass of a specific isotope to the total mass of all isotopes of the element present in a sample, reflecting its relative abundance. \\\\
    $Y$ & Fraction of helium: It is the proportion of helium in a mixture or celestial body. \\\\
    $Y_e$ & Average electron-to-baryon ratio: It is the average number of electrons per baryon in a substance. \\\\
    \resetinlineenum
    $Z$ & \inlineitem Metallicity: It is the proportion of mass in a celestial object that is attributable to elements heavier than helium, indicating the object's chemical composition and evolutionary history; \inlineitem Atomic number: It is the number of protons found in the nucleus of an atom, serving as a unique identifier for chemical elements and determining their position in the periodic table. \\\\
    $\bar{\rho}$ & Average density: It is the mean density of an object, calculated as total mass divided by total volume. \\\\
    $\dot{M}$ & Mass loss rate: It measures the rate at which mass is lost from a star. \\\\
    $\epsilon_F$ & Fermi energy: It is the highest energy level occupied by a fermion at absolute zero. \\\\
    $\Gamma$, $\gamma$ & Polytropic index: It is the exponent in a polytropic equation that quantifies the relationship between state variables in a variety of thermodynamic processes. \\\\
    \resetinlineenum
    $\mu$ & \inlineitem Mean molecular weight: It is the weighted average mass of the molecules in a substance, typically expressed in atomic mass units, accounting for the relative abundance of each molecular species; \inlineitem Chemical potential: It is a thermodynamic quantity representing the change in a system's energy with respect to the change in the number of particles, at constant temperature and volume, indicative of a particle's contribution to the overall system's energy; \inlineitem Reduced mass: It refers to the product of the two masses divided by their sum; \inlineitem Magnetic dipole moment: It represents the strength and orientation of a magnet or current loop's magnetic field, characterizing the torque it experiences in an external magnetic field. \\\\
    \resetinlineenum
    $\nu$ & \inlineitem Frequency: It is the number of occurrences of a repeating event per unit time; \inlineitem Neutrino: A fundamental, electrically neutral particle with very low mass, participating in weak interactions and capable of passing through matter almost undisturbed. \\\\
    \resetinlineenum 
    $\Omega$ & \inlineitem Orbital angular velocity: It is the rate at which an object orbits around another object or point in space; \inlineitem Spin frequency: It is the rate at which an object rotates around its own axis. \\\\
    $\Phi$ & Gravitational potential: It is the potential energy per unit mass at a point in a gravitational field. \\\\
    $\rho$ & Density: It is the mass per unit volume of a substance. \\\\
    $\tau_\mathrm{dyn}$, $\tau_\mathrm{ff}$ & Dynamical or free-fall timescale: The time it takes for an object to collapse under its own gravitational attraction, assuming no other forces act upon it. \\\\
    $\tau_\mathrm{nuc}$ & Nuclear timescale: The duration over which a star generates energy through nuclear fusion at its core. \\\\
    $\tau_\mathrm{therm}$, $\tau_\mathrm{KH}$ & Thermal or Kelvin-Helmholtz timescale: The time it takes for a star or gas giant to radiate away its gravitational binding energy at its current luminosity. \\\\
\end{longtable}


\textbf{Gradients}
\begin{longtable}{p{0.1\textwidth} p{0.8\textwidth}}
    $\nabla_\mathrm{rad},\,\nabla_\mathrm{ad},\,\nabla_\mathrm{\mu}$ & Gradients (Radiative, Adiabatic, Molecular): These represent the rates of change of temperature, pressure, and composition within a stellar interior, respectively. \\
\end{longtable}

\textbf{General Relativity and Cosmology}
\begin{longtable}{p{0.1\textwidth} p{0.8\textwidth}}
    $\eta_{\mu \nu}$ & Minkowski metric: The metric tensor of flat spacetime, used in special relativity to describe spacetime in the absence of gravitational fields. \\\\
    $\Lambda$ & Cosmological constant: A term in Einstein's field equations of general relativity that represents the density of energy in the vacuum of space, potentially responsible for the acceleration of the universe's expansion. \\\\
    $ds^2$ & Spacetime interval: An element of spacetime distance that is invariant under Lorentz transformations, fundamental in the theory of relativity to describe the interval between two events in spacetime. \\\\
    $dx^\alpha$ & Infinitesimal displacement vector in spacetime: Represents a small change in the four spacetime coordinates, used in the formulation of spacetime dynamics. \\\\
    $G_{\mu \nu}$ & Einstein tensor: Describes the curvature of spacetime caused by mass and energy. \\\\
    $g_{\mu \nu}$ & Metric tensor: Describes the geometry of spacetime and defines how distances are measured. \\\\
    $R$ & Scalar curvature: Gives a single scalar value that describes the curvature of a space at a point. \\\\
    $R_{\mu \nu}$ & Ricci tensor: Represents the amount by which the volume of a geodesic ball in spacetime deviates from that in Euclidean space due to curvature. \\\\
    $T_{\mu \nu}$ & Stress-energy tensor: Describes the density and flux of energy and momentum in spacetime, serving as the source of gravitational fields in Einstein's field equations.
\end{longtable}

\section*{Numerical Constants}
\begin{longtable}{p{0.1\textwidth} p{0.8\textwidth}}
    $e$ & $\approx 2.71828$: The base of natural logarithms. \\\\
    $i$ & $= \sqrt{-1}$: The imaginary unit. \\\\
    $\ln(10)$ & $\approx 2.30259$: The natural logarithm of 10, useful in converting between natural and common logarithms. \\\\
    $\log_{10}(e)$ & $\approx 0.43429$: The logarithm base 10 of \(e\), crucial for changing the base of logarithms from \(e\) to 10. \\\\
    $1\,\text{rad}$ & $= 57.2958\,\text{degrees}$: Defines the radian measure, crucial in angular measurements and conversions. \\\\
    $\phi$ & $\approx 1.61803$: The golden ratio, a number often appearing in patterns in nature and designs, including spiral galaxies. \\\\
    $\pi$ & $\approx 3.14159$: The ratio of the circumference of a circle to its diameter, essential in circular and orbital calculations. \\\\
\end{longtable}


\section*{Physical Constants}
\begin{longtable}{p{0.3\textwidth} p{0.6\textwidth}}
    Boltzmann constant & $k_B = 1.380649 \times 10^{-23}\ \text{J K}^{-1}$: The physical constant relating the average kinetic energy of particles in a gas to the temperature of the gas. \\\\
    Coulomb constant & $k_e = 8.9875517873681764 \times 10^{9}\ \text{N m}^2\text{C}^{-2}$: The fundamental physical constant quantifying the electrostatic force of attraction or repulsion between two charged particles. \\\\
    Electron mass & $m_e = 9.10938356 \times 10^{-31}\ \text{kg}$: The mass of an electron, characterizing the lightest stable charged lepton. \\\\
    Gravitational constant & $G = 6.67430 \times 10^{-11}\ \text{m}^3\text{kg}^{-1}\text{s}^{-2}$: The constant of proportionality in Newton's law of universal gravitation. \\\\
    Planck constant & $h = 6.62607015 \times 10^{-34}\ \text{m}^2\text{kg s}^{-1}$: Quantifies the amount of action or quantum of electromagnetic action linking photon energy with its electromagnetic wave frequency. \\\\
    Proton (hydrogen) mass & $m_H = 1.6726219 \times 10^{-27}\ \text{kg}$: Defines the mass unit for atoms and subatomic particles, representing the mass of a proton. \\\\
    Speed of light & $c = 2.99792458 \times 10^{8}\ \text{m s}^{-1}$: The constant speed at which all electromagnetic waves propagate in a vacuum. \\\\
    Stefan-Boltzmann constant & $\sigma = 5.670374419 \times 10^{-8}\ \text{W m}^{-2}\text{K}^{-4}$: Describes the power radiated from a black body in terms of its temperature.
\end{longtable}



\section*{Astronomical Constants}
\begin{longtable}{p{0.3\textwidth} p{0.6\textwidth}}
    Astronomical Unit ($1\,\text{AU}$) & $= 1.496 \times 10^{13}\ \text{cm}$: The average distance from the Earth to the Sun, used as a common unit of distance in astronomy. \\\\
    Light Year ($1\,\text{ly}$) & $= 9.461 \times 10^{17}\ \text{cm}$: The distance that light travels in vacuum in one Julian year (365.25 days), used to express astronomical distances. \\\\
    Parsec ($1\,\text{pc}$) & $= 3.086 \times 10^{18}\ \text{cm}$: The distance at which one astronomical unit subtends an angle of one arcsecond, commonly used in astronomy to measure distances to stars and galaxies beyond the Solar System. \\\\
    Solar Effective Temperature ($T_\mathrm{eff},\odot$) & $= 5,778\ \text{K}$: The temperature of a black body that would emit the same total amount of electromagnetic radiation as the Sun, used to approximate the surface temperature of stars. \\\\
    Solar Luminosity ($L_{\odot}$) & $= 3.828 \times 10^{33}\ \text{erg s}^{-1}$: The total amount of energy emitted by the Sun per second, used as a benchmark for the luminosity of other stars. \\\\
    Solar Mass ($M_{\odot}$) & $= 1.989 \times 10^{33}\ \text{g}$: The mass of the Sun, used as a standard unit of mass in astronomy to describe the mass of other stars and galaxies. \\\\
    Solar Radius ($R_{\odot}$) & $= 6.96 \times 10^{10}\ \text{cm}$: The radius of the Sun, serving as a common measure for comparing the size of stars. \\\\
\end{longtable}


\end{document}